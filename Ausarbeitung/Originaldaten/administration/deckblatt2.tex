%% +++++++++++++++++++++++++++++++++++++++++ %%
%                                             %
%     Vollständigkeit der Aufgabenstellung    %
%                                             %
%% +++++++++++++++++++++++++++++++++++++++++ %%

% 07.07.2017 Timo von Wysocki

%% Beinhaltet Seite 3 der offiziellen Anfangsseiten mit der Erklärung zur Vollständigkeit der Aufgabenstellung. Der Inhalt wird in der Datei ./inhalt/0000_persoenliche_daten definiert!

\titletemplate{		
Hiermit wird die Vollständigkeit der Aufgabenstellung bestätigt. Alle darüber hinaus gehenden Aufgaben sind nicht Teil der Abschlussarbeit oder werden in der Ausarbeitung als solche kenntlich gemacht. Sollte es der/dem Studierenden, ohne eigenes Verschulden, nicht möglich sein, diese Aufgabenstellung in der vorgesehenen Bearbeitungszeit vollständig zu erfüllen, ist dies in der Ausarbeitung zu begründen.
Mit ihrer/seiner Unterschrift nimmt die/der Studierende  die Aufgabenstellung an. 
Der Ausgabe- sowie Abgabetag ist somit bindend
	
%Die Diplomarbeit ist im engen Kontakt mit dem Institut auszuarbeiten und bleibt dessen Eigentum. Eine Einsichtnahme Dritter darf nur nach Rücksprache mit dem Institut erfolgen.

\vspace{2cm}

\centering
\begin{tabular}{@{}p{0.4\textwidth}p{0.1\textwidth}p{0.4\textwidth}@{}}
	Ausgabetag: \timestart &&Abgabetag: \timeend\\
	&&\\
	Betreuer:&&Projektleiter:\\
	&&\\
	&&\\
	&&\\	\cdashline{1-1}\cdashline{3-3}
	(Prof. Dr.rer.nat. Frank Gauterin)&&(\projektleiter)\\
	&&\\
	&&\\
	&&\\
	Bearbeiter:&&Anschrift:\\
	&&\\
	&& \mystreet\\
	&& \mytown\\ \cdashline{1-1}
	(\mydegree{} \myname)&&Tel.: \mytel\\
\end{tabular}

}{}
