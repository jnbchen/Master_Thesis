%% +++++++++++++++++++++++++++++++++++++++++ %%
%                                             %
%           Template für KIT-Rahmen           %
%                                             %
%% +++++++++++++++++++++++++++++++++++++++++ %%

% 07.07.2017 Timo von Wysocki

%% Erstellt den Ramen für die ersten vier Seiten inklusive Logos, Internetadressen und Anschriften. Zur Benutzung \titletemplate{Inhalt1}{Inhalt2}. Inhalt 1 wird korrekt in der Schriftfläche der Seite platziert. Inhalt 2 kann eigene Platzierungen oder Modifikationen enthalten.

% Abstände definieren:
\newcommand{\diameter}{25pt}		% Durchmesser des Bogens des Rahmens
\newcommand{\border}{19mm}			% Abstand des Rahmens zum Rand in mm
\newcommand{\logoborder}{7mm}		% Abstand des Logos zum Rand in mm
\newcommand{\titleBorder}{60mm} 	% Abstand des Textes zum Rand nach oben in mm
\newcommand{\imageBorder}{133mm}	% Abstand des Bildes zum Rand nach oben in mm
\newcommand{\reviewerBorder}{45mm}	% Abstand des Projektleiters zum Rand nach unten in mm
\newcommand{\webBorder}{5mm}		% Abstand der Website zum unteren Rand nach oben in mm

% Berechnen der Länge des Institutsnamens um Box zu erzeugen mit exakt dieser Breite um rechts ausgerichteten linksbündigen Text zu produzieren
\newlength{\institutsname}
\settowidth{\institutsname}{Institute of Vehicle Systems Technology} % Berechne die Breite diess Textes, für die ordentliche Ausrichtung des Adressblocks

% Neues Kommando mit 2 Inputparametern. Der erste wird in eine fertige Box geschrieben und platziert, im zweiten können beliebige Kommandos eingefügt werden, auch neue Boxen etc.
\newcommand{\titletemplate}[2]{
\begin{titlepage}	% Definiere eine Titlepage
	
% KIT-Rahmen um die Titlepage
\ifFrame
	\definecolor{FrameColor}{rgb}{0.5,0.5,0.5}
\else
	\definecolor{FrameColor}{rgb}{1,1,1}
\fi	

\begin{tikzpicture}[remember picture, overlay]
	\draw [color=FrameColor] 
	($(current page.north west)+(\border,-\border)$) -- % gerade linie von oben links 
	($(current page.north east)+(-\border-\diameter, -\border)$) % nach oben recht bis vor den Bogen
	arc (90:0:\diameter) -- % Bogen
	($(current page.south east)+(-\border, \border)$) % Linie nach unten rechts
	-- ($(current page.south west)+(\border+\diameter,\border)$) % Linie nach unten Link bis vor den Bogen
	arc (90:0:-\diameter) -- % Bogen unten links
	cycle;	% schließe den Bogen	
	\end{tikzpicture}

	
	


% KIT-Logo:
\begin{textblock*}{10cm}(\border+ \logoborder ,\border+\logoborder)
	\includegraphics[width=3.8cm]{./administration/grafiken/KIT.pdf}
\end{textblock*}
	
\ifLogo	
% Institutslogo:
\begin{textblock*}{10cm}[1,0](\paperwidth-\border-\logoborder,\border+\logoborder)
	\raggedleft
	\includegraphics[width=4.8cm]{./administration/grafiken/institut_fast.png}
\end{textblock*}
\else
% Institutsadresse	
\begin{textblock*}{\institutsname}	[1,0](\paperwidth-\border-\logoborder,\border+\logoborder) % Ausrichten an Ecke rechts (1) oben (0) der Box	
		\small{\textbf{Institute of Vehicle Systems Technology}}\\
		\small{\textbf{Devision of Vehicle Technology}}\\
		\footnotesize{Head: Prof. Dr. rer. nat. Frank Gauterin}\\
		\footnotesize{Rintheimer Querallee 2}\\
		\footnotesize{76131 Karlsruhe}
\end{textblock*}	
\fi

% Userinput #1:
\begin{textblock*}{\textwidth}[0.5,0](\paperwidth/2,\titleBorder)
#1
\end{textblock*}

% Userinput #2:
#2

% KIT-Schriftzug unter Rahmen:
\ifFrame
\begin{textblock*}{10cm}[0,0.5](\border+\logoborder,\paperheight-\border+\webBorder)
\tiny{ 
	\iflanguage{english}
		{KIT -- The Research University in the Helmholtz Association}
		{KIT -- Die Forschungsuniversität in der Helmholtz-Gemeinschaft}
}
\end{textblock*}
\fi

% KIT-WEbsite unter Rahmen
\ifFrame
\begin{textblock*}{5cm}[1,0.5](\paperwidth-\border-\logoborder,\paperheight-\border+\webBorder)
	\raggedleft
	% Versuche, den Link klickbar zu machen, ohne optischen Rahmen
	%\hypersetup{hidelinks=true}
	%\href{www.kit.edu}{\large{	\textbf{www.kit.edu}}}
	%\hypersetup{hidelinks=false}
	\large{	\textbf{www.kit.edu}}	
\end{textblock*}
\fi

\end{titlepage}}