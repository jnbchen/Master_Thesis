%% +++++++++++++++++++++++++++++++++++++++++ %%
%                                             %
%            Kurzfassung der Arbeit           %
%                                             %
%% +++++++++++++++++++++++++++++++++++++++++ %%

%% Seite 5 der Arbeit

\addchap{Kurzfassung}
\thispagestyle{empty}
\section*{Simulation des Fahrzeugmodells zur Erkennung von Straßenmerkmalen}

% Hier Text einfügen:

Der Zustand der Straßeninfrastruktur beeinflusst Wohlstand, Produktivität, Wachstum und soziales Wohlergehen moderner Volkswirtschaften.
%
Die gegenwärtige Praxis der Straßenzustandsüberwachung ist mühsam und zeitaufwendig, da die meisten Schritte des Prozesses manuell durchgeführt werden.
%
Nur wenige Länder verfügen über die teure Spezialfahrzeuge für eine automatisierte Datenerhebung.
%
Außerdem die Überwachung ist nur in festen Intervallen von 1 bis 4 Jahren stattgefunden.
%
Mit der gegenwärtigen Praxis ist es unmöglich, dass Defekte in die frühe Phase umfassend identifiziert werden, bis die Reparaturen kosteneffizienter sind.
%
Vision-basierte Systeme können nur dann sehr umfassende Informationen liefern, wenn die Sichtlinie nicht behindert wird und genügend Rechenressourcen verfügbar sind.
%
Daher wurden neue Verfahren auf der Basis von Fahrzeugsensoren, insbesondere ein Inertialsensor, in der Fahrzeugkarosserie und überwachtes maschinelles Lernen entwickelt, um die Straßeninfrastruktur automatisch zu überwachen.
%
Überwachte maschinelle Lerntechniken beinhalten jedoch eine kostspielige und mühsame Sammlung von gelabelten Daten zum Trainieren und Testen.
%
Wir haben eine neuartige Simulationsumgebung entwickelt, um diese Kosten zu reduzieren und spezifische Fahrzeugparameter und Einstellungen für die Klassifikationsergebnisse zu untersuchen und zu quantifizieren.
%
Die entwickelten Simulationen und Ergebnisse ist hilflich für die Verbesserung von dem Modell um den realen Anwendungen.