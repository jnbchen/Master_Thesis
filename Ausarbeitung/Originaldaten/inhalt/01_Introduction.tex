\chapter{Introduction}
\label{intro}

The condition of the road infrastructure has severe impacts on the driving comfort, driving driving safety, tire road noise and rolling resistance~\cite{ihs_influence_2005, sandberg_road_1987-1, descornet_road-surface_1990, masino_identification_2017}.
%
Subsequently, tire road noise leads to sleep disturbances, general annoyance, or speech interference~\cite{ohrstrom_effects_2006, schwela_world_2000}. 
%
Therefore the costs for noise abatement measures increases~\cite{bmvi_statistik_2015} and the value of houses and land close to noisy transport roads decreases~\cite{taylor_effect_1982, wilhelmsson_impact_2000, Theebe2004}.
%
A higher rolling resistance leads to higher vehicle operational costs, e.g. for fuel, and decreases the range of electric vehicles~\cite{barrand2008reducing}.
%
Furthermore, a delayed detection and maintenance of road damages results in the erosion of the road substance. Subsequently, a complete renewal increases the costs over the life cycle of the road and yields to a complete roadblock and traffic jams~\cite{forschungsgesellschaft_fur_strassen-_und_verkehrswesen_merkblatt_2004}. 
%
State departments of transport around the world mainly monitor the road infrastructure manually, which is laborious and time-consuming~\cite{radopoulou_improving_2016}.
%
Inspectors collect data manually and rate road segments based on their experience.
%
Only few countries employ specialized vehicles with an automated data collection~\cite{koch2011pothole}. 
%
However, they are sophisticated and expensive to use~\cite{radopoulou_improving_2016}.

Researches have addressed the common problem of road infrastructure monitoring and present methods, which are based on vehicle sensors and machine learning. 
%
The idea is to collect data with a crowd of vehicles with on-board sensors, which sense the environment of the vehicle~\cite{radopoulou_automated_2016}.A literature review about this approach is given in~\cite{radopoulou_automated_2016}.
%
The most promising sensors are cameras, such as the Bosch stereo camera, or inertial sensors in the vehicle body.

Novel developed computer vision techniques lead to good results of the detection of road defects.
%
However, only high class vehicles with expensive packages, e.g. magic body control from Mercedes-Benz, have a camera on-board.
%
Moreover, they are limited to decent weather conditions and higher velocities of the vehicle have a negative impact on the precision~\cite{radopoulou_automated_2016}.
%
Furthermore, computer vision needs a lot of storage and computer power since a large amount of data need to be processed.

In contrast to cameras, the inertial sensors are less computational intensive and independent from light conditions.
%
Several studies have shown that inertial sensor data from the vehicle body with data processing based on machine learning are suitable to detect road damages or estimate the road roughness~\cite{eriksson_pothole_2008, Chen.2013, ngwangwa_reconstruction_2010, nitsche_comparison_2012, seraj_roads:_2014}.
%
Most papers implement a classification, which predicts specific road defects, such as potholes.
%
The classifier is trained with inertial sensor data labeled with the desired output, here the road defects.
%
After the training process, the classifier is tested with new and unseen labeled data.
%
The output of the training and testing process are confusion matrices, from which performance measures can be derived, e.g. the accuracy, precision and recall.
%
However, only few kilometres and specific road segments with few variations of the driving conditions are applied in most papers.
%
More importantly, no study has performed a sensitivity analysis and has investigated the impact of different influences on the classification accuracy, e.g. the vehicle load, noise of the sensor, or the type of vehicle.

This research cap is closed and a novel simulation environment is presented, which is based on a full car model. The road features, such as potholes, railroad crossing are simulated with functions and those contact with tires is also considered.
% 
In this thesis, a basic full car model is enhanced to analyze the impact of different settings of the vehicle on the classification results of road defects.
%
The road is modeled as a surface with different degrees of roughness and the \ac{SPD} of our road corresponds to the \ac{SPD} of real road surfaces.
%
Furthermore, the position of the inertial sensor in the vehicle is varied in the simulation to identify the position with highest signal-to-noise ratio and which leads to the best classification accuracy.

The advantage of this novel environmental simulation in contrast to real test drives is the reproducibility and the precise investigation of specific influences.
%
Furthermore, the costs to perform this analysis is drastically reduced.
%
The results of our investigations helps the further development of automatic road condition monitoring system and reduces the efforts to investigate further influences or to test novel classification algorithms.