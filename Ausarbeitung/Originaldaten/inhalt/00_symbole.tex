%% +++++++++++++++++++++++++++++++++++++++++ %%
%                                             %
%               Symbolverzeichnis             %
%                                             %
%% +++++++++++++++++++++++++++++++++++++++++ %%

\chapter{Symbolverzeichnis}

% Unterverzeichnisse (z.B. für lateinische und griechische Buchstaben können als Section eingefügt werden)

% \todo{Ist ein Einheitsvektor einheitslos?}
\begin{longtable}[c]{@{}p{0.15\textwidth}p{0.6\textwidth}p{0.15\textwidth}@{}}
\caption{Lateinische Symbole\label{tab:symboleLateinisch}} \\
\toprule
Symbol & Bedeutung & Einheit\\
\midrule \endfirsthead
\caption{Lateinische Symbole -- Fortsetzung} \\
\toprule
Symbol & Bedeutung & Einheit\\
\midrule \endhead
\midrule
\multicolumn{3}{c}{\vdots}\\
\bottomrule \endfoot
\bottomrule \endlastfoot
$A_m$	&	Amplitude laufende Welle	&	\si{\milli\metre}\\
$A_s$	&	Amplitude stehende Welle	&	\si{\milli\metre}\\
$\ve{e}_r$&	Einheitsvektor in Radialrichtung	&	\si{\none}\\
$\ve{e}_t$&	Einheitsvektor in Tangentialrichtung	&	\si{\none}\\
$\ve{e}_x$&	Einheitsvektor in $x$-Richtung	&	\si{\none}\\
$\ve{e}_y$&	Einheitsvektor in $y$-Richtung	&	\si{\none}\\
$\ve{e}_z$&	Einheitsvektor in $z$-Richtung	&	\si{\none}\\
$f_m$	&	Frequenz laufende Welle		&	\si{\hertz}\\
$f_s$	&	Frequenz stehende Welle		&	\si{\hertz}\\
$n_m$	&	Anzahl Wellenbäuche über Umfang laufende Welle	&	\si{\none}\\
$n_s$	&	Anzahl Wellenbäuche über Umfang stehende Welle	&	\si{\none}\\
$r_{dyn}$&	Dynamischer Reifenhalbmesser	&	\si{\metre}\\
$t$		&	Zeit						&	\si{\second}\\
$v_r$	&	Rotationsgeschwindigkeit Oberfläche Reifen	&	\si{\metre\per\second}\\
%$\mu$ 	& Reibungskoeffizient 		& $[-]$ \\
%$A$ 	& Querschnitt/Profilfläche 	& $[mm^2]$\\
%$A_0$ 	& Grundfläche 				& $[mm^2]$ \\
%$E$ 	& Elastizitätsmodul 		& $[N/mm^2]$\\
%$e$ 	& Empfindlichkeit 			& $[N/V]$\\
%$\epsilon$	& Dehnung 				& $[-]$ \\
%$F$		& Kraft 					& $[N]$ \\
%$F_x$	& Umfangskraft 				& $[N]$ \\
%$F_y$	& Seitenkraft				& $[N]$ \\
%$F_z$	& Radlast					& $[N]$ \\
%$F_g$	& Gewichtskraft				& $[N]$ \\
%$I$		& Trägheitsmoment			& $[mm^4]$ \\
%$k$		& k-Faktor (Proportionalitätsfaktor DMS) &$[-]$\\ 
%$l$		& Länge						& $[mm]$ \\
%$M$		& Moment					& $[Nm]$ \\
%$M_x$	& Moment um die x-Achse		& $[Nm]$ \\
%$M_z$	& Moment um die z-Achse		& $[Nm]$ \\
%$m$		& Masse						& $[kg]$ \\
%$m'$	& Masse pro Längeneinheit	& $[kg/mm]$ \\
%$R$		& Elektrischer Widerstand	& $[\Omega]$ \\
%$R_m$	& Zugfestigkeit				& $[N/mm^2]$ \\
%$R_{eS}$& Elastische Streckgrenze	& $[N/mm^2]$ \\
%$R_{p0,2}$& 0,2\%-Dehngrenze		& $[N/mm^2]$ \\
%$S$		& Sicherheitsfaktor			& $[-]$ \\
%$\sigma$& Mechanische Spannung		& $[N/mm^2]$ \\
%$U$		& Elektrische Spannung		& $[V]$ \\
%$U_x$	& Spannung der Messbrücken der Umfangskraft & $[V]$ \\
%$U_y$	& Spannung der Messbrücken der Seitenkraft & $[V]$ \\
%$U_z$	& Spannung der Messbrücken der Radlast & $[V]$ \\
%$W$		& Widerstandsmoment			& $[mm^3]$ \\
%Ab hier wiederholt &&\\
%$\mu$ 	& Reibungskoeffizient 		& $[-]$ \\
%$A$ 	& Querschnitt/Profilfläche 	& $[mm^2]$\\
%$A_0$ 	& Grundfläche 				& $[mm^2]$ \\
%$E$ 	& Elastizitätsmodul 		& $[N/mm^2]$\\
%$e$ 	& Empfindlichkeit 			& $[N/V]$\\
%$\epsilon$	& Dehnung 				& $[-]$ \\
%$F$		& Kraft 					& $[N]$ \\
%$F_x$	& Umfangskraft 				& $[N]$ \\
%$F_y$	& Seitenkraft				& $[N]$ \\
%$F_z$	& Radlast					& $[N]$ \\
%$F_g$	& Gewichtskraft				& $[N]$ \\
%$I$		& Trägheitsmoment			& $[mm^4]$ \\
%$k$		& k-Faktor (Proportionalitätsfaktor DMS) &$[-]$\\ 
%$l$		& Länge						& $[mm]$ \\
%$M$		& Moment					& $[Nm]$ \\
%$M_x$	& Moment um die x-Achse		& $[Nm]$ \\
%$M_z$	& Moment um die z-Achse		& $[Nm]$ \\
%$m$		& Masse						& $[kg]$ \\
%$m'$	& Masse pro Längeneinheit	& $[kg/mm]$ \\
%$R$		& Elektrischer Widerstand	& $[\Omega]$ \\
%$R_m$	& Zugfestigkeit				& $[N/mm^2]$ \\
%$R_{eS}$& Elastische Streckgrenze	& $[N/mm^2]$ \\
%$R_{p0,2}$& 0,2\%-Dehngrenze		& $[N/mm^2]$ \\
%$S$		& Sicherheitsfaktor			& $[-]$ \\
%$\sigma$& Mechanische Spannung		& $[N/mm^2]$ \\
%$U$		& Elektrische Spannung		& $[V]$ \\
%$U_x$	& Spannung der Messbrücken der Umfangskraft & $[V]$ \\
%$U_y$	& Spannung der Messbrücken der Seitenkraft & $[V]$ \\
%$U_z$	& Spannung der Messbrücken der Radlast & $[V]$ \\
%$W$		& Widerstandsmoment			& $[mm^3]$ \\
\end{longtable}

\begin{longtable}[c]{@{}p{0.15\textwidth}p{0.6\textwidth}p{0.15\textwidth}@{}}
	\caption{Griechische Symbole\label{tab:symboleGriechisch}} \\
	\toprule
	Symbol & Bedeutung & Einheit\\
	\midrule \endfirsthead
	\caption{Griechische Symbole -- Fortsetzung} \\
	\toprule
	Symbol & Bedeutung & Einheit\\
	\midrule \endhead
	\midrule
	\multicolumn{3}{c}{\vdots}\\
	\bottomrule \endfoot
	\bottomrule \endlastfoot
	$\varphi$	&	Winkel in Polarkoordinaten	&	\si{\radian}\\
	$\omega_r$	&	Winkelgeschwindigkeit Reifen	&	\si{\radian\per\second}\\
\end{longtable}