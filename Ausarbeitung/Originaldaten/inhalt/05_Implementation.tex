\chapter{Implementation}

Another feature of mechatronic systems is integrated digital information processing~\cite{isermann2007mechatronische}.
%
For this purpose some software is used to solve problems.
%
Consequently, the work on the employed software will be explained in this chapter.

\section{Introduction of the software}

\textsc{Matlab} is a numerical solution software environment for many technical and scientific problems.
%
It's suitable for quick analysis and synthesis dynamic processes in research and industry.
%
Meanwhile it has become the standard tool for many types of technical calculations as well as simulations in system dynamics, control engineering, aviation and vehicle developing.
%

\textsc{Matlab} comes with a basic set of tools for visualizing data and for performing calculations on matrices and vectors.
%
For specific technologies, it provides toolboxes, which add to the basic functionality.
%
The toolboxes used in this work are

\begin{itemize}
\item Control System Toolbox™
\item Signal Processing Toolbox™
\item Robust Control Toolbox™ 
\item \textsc{ScixMiner} (available under \url{https://sourceforge.net/projects/scixminer/})
\end{itemize}

Control System Toolbox™ provides algorithms and apps for systematically analyzing, designing, and tuning linear control systems.
%
We can specify the system as a transfer function, state-space, zero-pole-gain, or frequency-response model and analyze, visualize the system behavior in the time and frequency domains~\cite{Mathworks_cst}.

Signal Processing Toolbox™ provides functions and apps to generate, measure, transform, filter, and visualize signals.
%
The toolbox includes algorithms for resampling, smoothing, and synchronizing signals, designing and analyzing filters, estimating power spectra, and measuring peaks, bandwidth, and distortion.
%
We can use Signal Processing Toolbox to analyze and compare signals in time, frequency, and time-frequency domains, identify patterns and trends, extract features, and develop and validate custom algorithms to gain insight into the data~\cite{Mathworks_spt}.

Robust Control Toolbox™ provides functions and blocks for analyzing and tuning control systems for performance and robustness in the presence of plant uncertainty.
%
We can analyze the impact of plant model uncertainty on control system performance, and identify worst-case combinations of uncertain elements.
%
H-infinity and mu-synthesis techniques can maximize the robust stability and performance of the controllers~\cite{Mathworks}.

\section{Explanation of the scripts in \textsc{Matlab}}

Here the basic part and the used commands in the script will be explained.
%
The version of \textsc{Matlab} which is used for the thesis is 2016a.
%
The details are hidden by abridgments due to the limit of the space.

The road model is wrote as a function which can create a road profile with the given roughness and random number of obstacles that are distributed on the random position on the road.

\begin{lstlisting}[caption={Function of the road model},label=code:road]
function profile = road_model(typ)
...
% PSD is a function for creating a random road.
% road is an empty matrix for storing the value of the desired road.
roughness = PSD(IRI,L);
road = zeros(size(roughness));
...
% cut the road into 100 pieces for inserting obstacles
interval = L/100/dr;
% random number of the obstacles (from 5 to 20)
n = randi([5 20],1);
% random position of the n obstacles on the road
position = randi([1 100],[1 n]);
% random gain (0.5 to 2.5) for shape of the obstacle
k = 0.5+2*rand(2,n);
...
for i = 1:n
    if strcmp(typ,'pothole')
        % a/b is the length/height of the obstacle which are modified by k
        a = k(1,n)*0.3;
        b = k(2,n)*0.05;
        % function of the pothole
        obstacle = pothole(a,b);
    elseif strcmp(typ,'manhole cover')
        ...
    end
    % insert the obstacle at the corresponding position
    road(interval*(n-1)+1:interval*(n-1)+length(obstacle)) = obstacle;
end
% add the roughness
road = road + roughness;
...
end

\end{lstlisting}

where 'typ' is the type of the obstacle including 'pothole', 'manhole\_cover', 'cobbled\_road', 'railway\_crossing' etc.
%
'dr' is the sampling length of the road. 
%
'PSD' and 'pothole' are the function of the random road which are explained in Sec.~\ref{sec:roadmodel}
%
The output 'profile' is a matrix with two rows which store the information of the road profiles under the left and right wheels separately.

The \ac{FCM} is also wrote as a function.
%
Here the road model which contains the road profiles of both side is as the input of the function.
%
The output of the function is the desired accelerations such as $\ddot{z}$, $\ddot{\phi}$ and $\ddot{\theta}$ etc.

\begin{lstlisting}[caption={Function of the \ac{FCM}},label=code:fcm]
function Output = FCM_passive(road_left,road_right,V,auto)
...
if strcmp(auto,'BMW_116d')
    % body mass
    Ms=1400;
    % tire spring of front-right
    Kwr1=160000;
    % suspension damper of front-right
    Csr1=1000;
    ...
elseif strcmp(auto,'S-Klasse_W220')
    ...
end
...
% sampling time
dt = dr/V;
% running time of the simulation
tmax = L/V;
% time array
t = 0:dt:tmax; 
% length of the car
car = round((Lf+Lr)/dr);
% front axle is one car ahead of the rear axle
left1 = road_left(car+1:end);
right1 = road_right(car+1:end);
left2 = road_left(1:end-car);
right2 = road_right(1:end-car);
...
% to implement the dynamic system as a state space model
fcm = ss(A,B,C,D);
% input is the road profile of each wheel
input = [right1;left1;right2;left2];
% solve the system with the input in time 't'
output = lsim(fcm,input,t);
...
end

\end{lstlisting}

where 'V' is the setting velocity of the vehicle.
%
'auto' is the model of the selected vehicle.
%
'A', 'B', 'C' and 'D' are the Matrices of the state space which is introduced in Sec.~\ref{sec:full car model}.
%
The command 'ss' and 'lsim' in the Control System Toolbox™ are used to create and solve the dynamic system.

Besides the road model and \ac{FCM} there are also several commands that will be used for the modelling and data processing.

To design a filter for the feature extraction in Sec.~\ref{sec:feature_ex} and the weighting function in Sec.~\ref{sec:active full car model}, the Butterworth filer is chosen for its flat pass bands and wide transition bands.

\begin{lstlisting}[caption={Butterworth filter design in Matlab},label=code:butter]
[b,a] = butter(n,Wn,ftype)
\end{lstlisting}

It designs a lowpass, highpass, bandpass, or bandstop Butterworth filter, depending on the value of 'ftype' and the number of elements of ‘Wn'.
%
The resulting bandpass and bandstop designs are of order $2n$.
%
'b' and 'a' are the transfer function coefficients of the filter.

To build the model of the \ac{FCM} with active suspension, the command below is used in Sec.~\ref{sec:active full car model} to compute q stabilizing $H\infty$ optimal controller.

\begin{lstlisting}[caption={$H\infty$ controller design in Matlab},label=code:h8]
K = hinfsyn(P,NMEAS,NCON)
\end{lstlisting}

where 'K' is the desired controller.
%
'P' is the plant including the weighting functions.
%
'NMEAS' and 'NCON' are the numbers of control signals and measurement signals.

\section{Working with \textsc{ScixMiner}}

\textsc{ScixMiner} bases on \textsc{Matlab} is designed for the visualization and analysis of time series and features with a special focus to data mining problems including classification, regression, and clustering~\cite{SciXMiner}.
%
It was developed at the Institute of Applied Computer Science of the Karlsruhe Institute of Technology.

The graphical user interface of \textsc{ScixMiner} contains menu items and control elements like listboxes, checkboxes and text fields.
%
\textsc{ScixMiner} use the project file $*.prjz$ as the import of data, which can be generated by the function
\texttt{generate\_new\_scixminer\_project}.

Apply the design with the appropriately setting of the features selection e.g. \ac{MANOVA}, number of the features and type of classifier, the classification will be executed.
%
This process can be also automatically completed by macros, which are files containing sequences of clicked menu items and control elements.

\begin{lstlisting}[caption={Marcos of the \textsc{ScixMiner}},label=code:scix]

%% training %%

% Selection of single features
% {'MANOVA'}
set_textauswahl_listbox(gaitfindobj('CE_Klassifikation_Merkmalsauswahl'),{'MANOVA'});eval(gaitfindobj_callback('CE_Klassifikation_Merkmalsauswahl'));

% Number of selected features
set(gaitfindobj('CE_Anzahl_Merkmale'),'string','15');eval(gaitfindobj_callback('CE_Anzahl_Merkmale'));

% Chosen classifier
% {'Support Vector Machine'}
set_textauswahl_listbox(gaitfindobj('CE_Klassifikation_Klassifikator'),{'Support Vector Machine'});eval(gaitfindobj_callback('CE_Klassifikation_Klassifikator'));

% Kernel order
% 1
set(gaitfindobj('CE_SVM_Ordnung'),'string','1');eval(gaitfindobj_callback('CE_SVM_Ordnung'));

% Graphical evaluation of classification results
set(gaitfindobj('CE_Anzeige_KlassiErg'),'value',1);eval(gaitfindobj_callback('CE_Anzeige_KlassiErg'));

% Save confusion matrix in file
set(gaitfindobj('CE_Konfusion_Datei'),'value',1);eval(gaitfindobj_callback('CE_Konfusion_Datei'));

%% Classification,  Data mining,  Design and apply 
eval(gaitfindobj_callback('MI_EMKlassi_EnAn'));

%% Data mining,  File,  Save classifier 
eval(gaitfindobj_callback('MI_Classifier_Export'));


%% testing %%

%% Data mining,  File,  Load classifier 
eval(gaitfindobj_callback('MI_Classifier_Import'));

%% Classification,  Data mining,  Apply 
eval(gaitfindobj_callback('MI_EMKlassi_An'));

\end{lstlisting}

In the macro, the method of the feature selection is set as \ac{MANOVA}, the number of the selected features is 15, the method of the classification is set as \ac{SVM} and the order of the kernel is one.

After the processing of the data, the results will be graphical evaluated and the confusion matrix will be saved in files.
Besides, the trained classifier will also be saved and be applied to the testing process.