%% +++++++++++++++++++++++++++++++++++++++++ %%
%                                             %
%             Abstract  der Arbeit            %
%                                             %
%% +++++++++++++++++++++++++++++++++++++++++ %%

%% Seite 6 der Arbeit

\addchap{Abstract}
\thispagestyle{empty}
\section*{Simulation of vehicle model for road features detection}

% Hier Text einfügen:
% \blindtext

The condition of the road infrastructure influences the prosperity, productivity, growth and social wellbeing of modern economies.
%
Current practice of road condition monitoring is laborious and time-consuming as most steps of the process are done manually.
%
Few countries have expensive specialized vehicles for an automated data collection whereas the monitoring is only scheduled in fixed intervals of 1 to 4 years.
%
With current practice defects are unlikely to be comprehensively identified in early stages, when repairs are more cost-efficient.
%
Vision-based systems can provide very comprehensive information only when the line of sight is not obstructed and enough computational resources are available.
%
Therefore, new methods based on vehicle sensors, especially an inertial sensor, in the vehicle body, and supervised machine learning have been developed to autonomously monitor the road infrastructure.
%
However, supervised machine learning techniques involve a costly and laboriously collection of labeled data for training and testing.
%
We have developed a novel simulation environment to reduce these costs and to investigate and quantify specific vehicle parameters and settings on the classification results.
%
The developed simulation and results help to improve the models for real world applications.
